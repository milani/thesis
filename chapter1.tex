\chapter{پیش‌زمینه}\markboth{پیش‌زمینه}{پیش‌زمینه}\label{chap:prerequisites}
\clearpage
\section{نظریه گراف}
در این بخش، نظریه گراف و نحوه نمادگذاری‌ را مرور خواهیم کرد. نمادگذاری ارائه شده در این بخش برای پیگیری بخش‌ها و فصول بعدی مورد نیاز است.

\subsection{گراف، زیرگراف و یکریختی}
یک \خمیده{گراف}، یک دوتایی \گراف{G} است که در آن
$V = \{v_{1},v_{2},...,v_{n}\}$
یک مجموعه مرتب شامل $n$ گره‌ یا رأس است و
$E \subseteq V\times V$
مجموعه یال‌هاست. اندازه گراف برابر اندازه مجموعه \V تعریف می‌شود که در اینجا $n$ است.

گراف \گراف{G}، یک گراف \خمیده{بدون جهت} است اگر برای هر دو رأس
$v_{i},v_{j} \in V$
، از اینکه
$\edge{v}{i}{j}{E}$
بتوان نتیجه گرفت که
$\edge{v}{j}{i}{E}$
؛ در غیر این صورت گراف \خمیده{جهت‌دار} است.

به هر یال به صورت \یال{i}{i}
یک \خمیده{دور} گفته می‌شود. بطور کلی بین دو رأس \Vi و \Vj در یک گراف، می‌تواند بیش از یک یال وجود داشته باشد. یک \خمیده{گراف ساده}، گرافی بدون دور است که بین هر دو رأس آن حداکثر یک یال وجود داشته باشد.

در این پایان‌نامه منظور از گراف، گراف ساده بدون جهت است؛ در غیر این صورت عبارت دقیق قید خواهد شد.

یک گراف ساده را می‌توان با \خمیده{ماتریس مجاورت} \A به اندازه $n\times n$نمایش داد. درایه $(i,j)$ از ماتریس \A برابر 1 است اگر یال \یال{i}{j} وجود داشته باشد. در غیر این صورت این درایه برابر صفر است. بدیهی است که ماتریس مجاورت یک گراف بدون جهت، متقارن است.

رئوس و/یا یال‌های یک گراف می‌توانند برچسب داشته باشند. یک گراف برچسب‌دار را با سه‌تایی $(V,E,l)$ نشان می‌دهیم که در آن $l: X \mapsto \Sigma$ تابعی است که یک برچسب از الفبای $\Sigma$ را به هر عضو مجموعه \X نسبت می‌دهد که \X می‌تواند بسته به اینکه رئوس یا یال‌ها و یا هر دو برچسب‌گذاری شده باشند به ترتیب برابر \V یا \E و یا $V\cup E$ باشد.

دو گراف \گراف{G} و
$\graph{G^{\prime}}{V^{\prime}}{E^{\prime}}$
\خمیده{یکریخت} (با نماد $G^{\prime} \simeq G$) هستند اگر تابع نگاشت دوسویی $f: V \rightarrow V^{\prime}$ (تابع یکریختی) وجود داشته باشد به طوری که
$(v_{i},v_{j}) \in E$
اگر و تنها اگر
$(f(v_{i}),f(v_{j})) \in E^{\prime}$
. اگر $G = G^{\prime}$ باشد آنگاه به $f$ تابع \خمیده{خودریختی} می‌گویند. برای دو گراف برچسب‌دار $G(V,E,l)$ و $G^{\prime} = (V^{\prime},E^{\prime},l^{\prime})$ تابع یکریختی (همچنین خودریختی) باید رابطه 
$l(v_{i}) = l^{\prime}(f(v_{i}))$
 را به ازای هر $v_{i} \in V$ نیز ارضاء کند.
 
به یک تابع با آرگومان از نوع گراف، ناوردای گرافی
\پانوشت{\متن‌لاتین{graph invariant}} گفته می‌شود اگر به دو گراف یکریخت مقدار یکسانی نسبت دهد. مثلاً توابع تعداد رئوس و تعداد یال‌ها توابع ناوردای گرافی می‌باشند.

برای دو گراف \گراف{G} و $\graph{G^{\prime}}{V^{\prime}}{E^{\prime}}$، می‌گوییم $G^{\prime}$ یک \خمیده{زیرگراف} از $G$ است (با نماد $G^\prime \subseteq G$) اگر $V^\prime \subseteq V$ و $E^\prime \subseteq E$. اگر $G^\prime \subseteq G$ و $E^\prime$ شامل تمام یال‌های $(u,v) \in E$ باشد به طوری که $u,v \in V^\prime$، آنگاه می‌گوییم $G^\prime$ یک \خمیده{زیرگراف القایی} $G$ است و آن را با نماد $G^\prime \sqsubseteq G$ نشان می‌دهیم.

\subsection{همسایگی و درجه}
دو رأس \Vi و \Vj از گراف $G$ \خمیده{مجاور}\پانوشت{adjacent} و یا \خمیده{همسایه}\پانوشت{neighbour} هستند اگر $(v_i,v_j) \in E$. برای رأس $v$ همسایگی $\mN(v)$، مجموعه‌ی رئوسی است که در مجاورت آن قرار دارند؛ به عبارتی
$\mN(v) = \{v_i | (v,v_i) \in E\}$.
به تمام یال‌ها به فرم $(v,v_i) \in E$ \خمیده{یال‌های حادث}\پانوشت{incident} رأس $v$ گفته می‌شود.

درجه یک رأس مثل $v$ که با نماد $d(v)$ نشان داده می‌شود، تعداد یال‌های حادث با آن رأس است. برای گراف‌های ساده بدون جهت عدد اصلی\پانوشت{cardinality} مجموعه $\mN(v)$ با $d(v)$ برابر است.

\subsection{گشت، مسیر، دور، زیردرخت}
یک \خمیده{گشت} در گراف، دنباله‌ای از رئوس است که دو رأس متوالی توسط یک یال به هم متصل شده‌باشند. یک \خمیده{مسیر} گشتی است که تمام رئوس آن غیرتکراری باشند. به یک مسیر بسته \خمیده{دور} گفته می‌شود. گراف G \خمیده{همبند} است اگر بین هر دو رأس آن یک مسیر وجود داشته باشد.

% به زیرگراف همبند $G^\prime = (V^\prime,E\prime)$ از گراف \گراف{G} یک \خمیده{مؤلفه همبندی} گفته می‌شود
 
یک زیردرخت (ریشه‌دار)، زیرگرافی بدون دور از یک گراف است
\section{مقایسه گراف‌ها}
\section{روش‌های نمایش گراف}
\section{گراف کرنل}
\subsection{کرنل‌ها در یادگیری ماشین}
\subsection{مبانی گراف کرنل‌}
\subsection{کرنل‌های مبتنی بر مسیر}
\subsection{کرنل‌های مبتنی بر زیرگراف‌های کوچک}
\subsection{کرنل‌های مبتنی بر زیردرخت‌ها}
