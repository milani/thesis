\chapter*{پیش‌گفتار}
\markboth{پیش‌گفتار}{پیش‌گفتار}

منطق شناختی احتمالاتی پویا منطقی نسبتاً جدید است که قبل از مطالعه‌ی آن می‌بایست منطق شناختی، منطق شناختی پویا و منطق شناختی احتمالاتی معرفی شده باشد و به فراخور در فصول سه‌گانه‌ به معرفی و توصیف هریک خواهیم پرداخت. 

مقدمه را با بهره‌جستن از مقدمه‌ی مقاله‌ی \citep{Kooi2003} نگاشتم و البته هر کجا که لازم بود از مقدمه‌های دیگر مقالات نیز استفاده کردم.

در فصل 1 ابتدا با کمک \citep{DELDitmarsch2007} و \citep{Sack2007} به خلاصه‌ای از منطق شناختی می‌پردازیم، سپس بر مبنای \citep{Fagin1994} منطق شناختی احتمالاتی را که همان منطق شناختی سنتی است به اضافه‌ی توانایی استدلال درباره‌ی احتمال معرفی کرده و تمامیت آن را اثبات می‌کنیم. 

فصل 2 بر اساس \citep{Benthem2009} نوشته شده است و به معرفی منطق‌های پویا اعم از شناختی پویا و شناختی پویای احتمالاتی اختصاص دارد. در این فصل پس از معرفی منطق اعلان عمومی و سپس گونه‌ای تا حدی تعمیم یافته از منطق شناختی پویا با در نظر گرفتن مدلی ساده‌ از منطق‌ شناختی احتمالاتی، منطق شناختی پویای احتمالاتی معرفی شده و تمامیت آن اثبات می‌شود. برای آنکه منطق‌های شناختی پویا را احتمالاتی کنیم ابتدا سه گونه‌ی طبیعی احتمال را تعریف می‌کنیم که عبارتند از: احتمال {\prior} جهان‌ها، احتمال رخداد عمل‌ها بر اساس فرایندی متناظر با دیدگاه عامل‌ها و احتمال خطا در مشاهده‌ی عمل‌ها. برای اثبات تمامیت منطق‌های پویا اصول موضوعه‌ای مطرح می‌شود تا بتوان معادل با هر فرمول در این منطق‌ها، با حذف عملگر پویا، فرمولی در منطق‌های ایستا بدست آورد. این اصول اثرات متقابل عملگر پویا با اتم‌ها و عملگرهای بولی و شناختی را توصیف می‌کنند. پس از اثبات صحت، با کمک آنها تمامیت منطق‌های پویا را از تمامیت منطق‌های ایستا نتیجه می‌گیریم.

هنگامی که به همراه ولی‌زاده به مطالعه و اثبات جزئیات مقاله‌ی \citep{Benthem2009} مشغول بودیم با مسائل و مشکلاتی برخورد کردیم که به تناسب در بخش‌های مختلف فصل ۲ با عنوان ملاحظه به آنها اشاره خواهم کرد.

فصل 3 نیز بر گرفته از \citep{Kooi2003} است و به منظور ارائه‌ی مثالی برای نشان دادن کاربرد منطق شناختی پویای احتمالاتی نگاشته شده است. مثالی که ارائه می‌شود معمای معروف \lr{Monty Hall} است که پیش از این مقاله راه‌حلی صوری برای آن داده نشده بود، ولی ما بر اساس این مقاله به راه‌حلی صوری با کمک گونه‌ای از منطق‌های شناختی پویای احتمالاتی دست می‌یابیم که به زیبایی جوابی معقول در اختیار می‌گذارد. این راه‌حل زیبا را نمی‌توانستم با جزئیات بیان کنم اگر از \citep{Kooithesise2003} استفاده نمی‌کردم و آن را نیافتم مگر با راهنمایی دکتر کویی.
