\clearpage
%\phantomsection
\addcontentsline{toc}{chapter}{مقدمه}
\chapter*{مقدمه}\markboth{مقدمه}{مقدمه}

امروزه با پیشرفت تکنولوژی و گسترش دانش، بشر با تولید حجم عظیمی از داده‌ها\پانوشت{data} روبرو است. شرکت‌های تجاری، مراکز علمی و مؤسسات تحقیقاتی سعی دارند از این داده‌ها در جهت برآوردن نیاز مشتریان، درک بهتر جهان هستی و ساخت تکنولوژی‌های جدید بهره ببرند. استفاده از این داده‌ها جز با استخراج دانش از آن‌ها ممکن نیست. طراحی روش و الگوریتم برای استنتاج قواعد کلی از یک سری داده‌ی مشاهده شده، موضوع اصلی یادگیری ماشین است.

داده یک مفهوم قراردادی و وابسته به دامنه\پانوشت{domain} است. از این‌رو برای نمایش آن، روش‌های مختلفی وجود دارد: در بعضی کاربردها، کافی است داده‌ها را به صورت عددی یا برداری\پانوشت{vector} از اعداد نمایش داد. اما در برخی موارد، نمایش سطح بالاتری برای بیان حق مطلب مورد نیاز است. برای مثال، اگر بخواهیم موضوع یک متن را بر اساس تکرار کلمات پیش‌بینی کنیم، کافی است یک بردار عددی تشکیل دهیم که عناصر این بردار نمایش‌دهنده میزان تکرار کلمات از پیش تعیین شده در آن متن است. اما اگر بخواهیم بین رفتار یک مولکول با ساختار آن رابطه‌ای برقرار کنیم، آنگاه نمی‌توان به سادگی مولکول را یک عدد یا برداری از اعداد در نظر گرفت. ساختار مولکول و همچنین بسیاری دیگر از داده‌ها معمولاً با بهره‌گیری از گراف‌ها مدل می‌شود.

گراف ابزاری ساده و انعطاف‌پذیر برای توصیف اشیاء و ارتباط بین آن‌هاست. یک گراف از یک مجموعه گره‌\پانوشت{node} و یک مجموعه یال‌\پانوشت{edge} که ارتباط بین گره‌ها را معین می‌کند، تشکیل می‌شود. معمولاً  گره‌ها، موجودیت‌ها و یال‌ها، ارتباط بین موجودیت‌ها را نمایش می‌دهند. به عنوان مثال، می‌توان یک شبکه اجتماعی را با استفاده از گراف مدل کرد: مجموعه‌ی گره‌ها، مجموعه‌ی اعضای این شبکه و هر یال بین دو گره، نشان‌دهنده‌ی ارتباط دو عضو متناظر در آن شبکه اجتماعی است؛ به همین ترتیب می‌توان یک مولکول را با یک گراف مدل کرد: مجموعه‌ی گره‌ها، مجموعه‌ی اتم‌های آن مولکول و هر یال بین دو گره نمایانگر وجود یک پیوند بین اتم‌های متناظر است. همانطور که از این دو مثال مشخص است گاهی موجودیت مورد مطالعه، یک گره در یک گراف است و ارتباط آن با دیگر عناصر در یک سیستم بزرگتر نمایش‌داده می‌شود و گاه کل یک گراف نمایشی از یک موجودیت است که در این حالت، گره‌ها و یال‌ها نمایش‌دهنده‌ی عناصر سازنده آن موجودیت هستند.

به دلیل همین انعطاف‌پذیری، گراف در بسیاری از رشته‌ها مورد توجه قرار گرفته است و بسیاری از انواع داده با گراف مدل می‌شوند. در زیر به کاربرد گراف در دو رشته که با موضوع پایان‌نامه پیشرو مرتبط است، اشاره می‌کنیم.

\درشت{شیمی} یکی از رشته‌هایی است که استفاده‌ی فراوانی از گراف دارد\جستار{Gasteiger_2003}. جستجو برای یافتن ترکیبات شیمیایی‌ که خواص مشخصی داشته باشند امری رایج در شیمی است. روش‌های آزمایشگاهی برای اینکار بسیار پر هزینه و زمانبر هستند. از اینرو روش‌های محاسباتی‌ای مورد نیاز است که بتواند چند ترکیب با خصوصیات مدنظر را به عنوان نامزد معرفی کند. در این صورت می‌توان با هزینه‌ی کمتر این نامزدها را در آزمایشگاه مورد سنجش قرار داد. برای این منظور، ترکیبات شیمیایی به صورت گراف-همانطور که در بالا شرح داده‌شد-مدل می‌شوند. در روش محاسباتی، که معمولاً با عنوان QSAR و QSPR شناخته می‌شود فرض این است که مولکول‌هایی که ساختار مشابه دارند دارای خواص مشابه هستند. بنابراین اندازه‌گیری میزان شباهت دو گراف مسئله‌ای مهم در شیمی محاسباتی\پانوشت{chemoinformatics} است.

\درشت{بیوانفورماتیک}\پانوشت{bioinformatics} شاخه نسبتاً جدیدی در زیست‌شناسی است که استفاده از گراف در آن رایج است\جستار{Junker_2008}. از گراف‌ برای نمایش داده‌هایی نظیر ساختار پروتئین‌\پانوشت{\متن‌لاتین{protein structure}}، هم‌بیانی\پانوشت{co-expression} پروتئین‌ها یا ژن‌ها، تعاملات پروتئین-پروتئین\پانوشت{\متن‌لاتین{protein-protein interactions}}، شبکه‌های تنظیمی\پانوشت{\متن‌لاتین{regulatory networks}} و شبکه‌های متابولیک\پانوشت{\متن‌لاتین{metabolic networks}} استفاده می‌شود. سؤالات جالبی در ارتباط با این گراف‌ها مطرح می‌شود. مثلاً ارتباط ساختار پروتئین با خصوصیات یا عملکرد\پانوشت{function} آن چیست، آیا می‌توان با استفاده از ساختار پروتئین‌ها تعاملات بین آن‌ها را پیش‌بینی کرد، شبکه‌های تنظیمی چطور بر روی رُخ‌نمود\پانوشت{phenotype} تاثیر دارند و سؤالاتی از این دست. تا به امروز، داده‌های زیستی به طور معمول ناکامل و نویزدار هستند و طبیعتاً گراف‌هایی که از آن‌ها بدست می‌آید تصویر درست و کاملی ارائه نمی‌کند. به همین دلیل استفاده از تکنیک‌های یادگیری ماشین در این رشته، پر کاربرد و در عین حال چالش برانگیز است.

علاوه بر این رشته‌ها، گراف‌ها در تحلیل شبکه‌های اجتماعی\جستار{Carrington_2005}، پردازش تصویر\جستار{Lezoray_2012} و تحلیل شبکه‌ جهانی وب مورد استفاده گسترده قرار می‌گیرند.

درست است که گراف تصویر روشن و ملموسی از داده بدست می‌دهد، اما این ویژگی هزینه‌ی خود را دارد: بسیاری از عمل‌ها\پانوشت{operations} که برای ساختارهایی نظیر آرایه و ماتریس بدیهی هستند، برای گراف تعریف نشده و یا پیچیده‌اند. برای مثال می‌توان به رابطه تساوی اشاره کرد. هیچ الگوریتمی برای تست برابری دو گراف وجود ندارد که در زمان خطی قابل اجرا باشد\جستار{Garey_1979}.

اکثر تکنیک‌های قوی در یادگیری ماشین با داده‌های برداری که نمایش دهنده‌ی نقاط در فضای اقلیدسی هستند، کار می‌کنند و به روشی برای اندازه‌گیری شباهت بین این نقاط نیاز دارند. به همین دلیل، تحقیقات روی گراف‌ها و یادگیری ماشین بیشتر به سمت نمایش گراف‌ها در قالب بردار در فضای اقلیدسی (و یا فضاهای برداری دیگر) گرایش دارد. هرچند نحوه تبدیل گراف به بردار یا ساختارهای دیگر، وابسته به موضوع مورد مطالعه است، اما طرح روشی که عمومیت و انعطاف‌پذیری داشته باشد مورد توجه قرار خواهد گرفت.

در این پایان‌نامه برآنیم که مشکلات روش‌های پیشین برای اندازه‌گیری میزان شباهت دو گراف را شرح دهیم و روش بهتری به همین منظور ارائه کنیم.

در فصل \ارجا{chap:prerequisites} بطور مختصر نظریه‌ی گراف، مقایسه گراف‌ها، روش تابع کرنل\پانوشت{\متن‌لاتین{kernel method}} در یادگیری ماشین و توابع کرنل گرافی\پانوشت{\متن‌لاتین{graph kernels}} را بررسی می‌کنیم. در فصل \ارجا{chap:gaussian-graphlet-kernel} تابع گرافلت کرنل گاوسی\پانوشت{\متن‌لاتین{gaussian graphlet kernel}} را معرفی می‌کنیم و دقت آن را در تشخیص مد‌ل‌های تصادفی گراف از یکدیگر، می‌سنجیم. در فصل \ارجا{chap:protein_function_prediction} توانایی تابع گرافلت کرنل گاوسی در پیش‌بینی عملکرد پروتئین‌ها را مورد مطالعه قرار می‌دهیم و نشان می‌دهیم که این کرنل در مقایسه با کرنل‌های پیشین، دقت بالاتری دارد. و در آخر، در فصل \ارجا{chap:network-completion-problem-ppi} از گرافلت کرنل گاوسی برای پیش‌بینی تعاملات پروتئینی و تصحیح تعاملات موجود، استفاده می‌کنیم. 

\subsection*{
مقالات مستخرج از این پایان نامه
}
از این پایان‌نامه یک مقاله تحت عنوان 
\lr{Protein function prediction via Gaussian graphlet kernel}
از فصل سوم، استخراج و برای داوری ارسال شده‌است. 
مقاله دیگری تحت عنوان 
\lr{Structure-based denoising of PPI networks}
برگرفته از فصل چهارم، نیز در حال آماده‌سازی است.