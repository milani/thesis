%%%%%%%%%%%%%%%%%%%%%%%%%%%%%%%%%%%%%%%%%
%صفحه‌ی بسم الله
%\thispagestyle{empty}
%\centerline{{\includegraphics[width=9 cm]{*}}}
%\newpage
%\thispagestyle{empty}
%\clearpage
%~~~
%%%%%%%%%%%%%%%%%%%%%%%%%%%%%%%%%%%%%%%%%
\clearpage
%\pagenumbering{adadi}

% دانشگاه خود را وارد کنید
\university{شهید بهشتی}
% دانشکده، آموزشکده و یا پژوهشکده  خود را وارد کنید
\faculty{علوم ریاضی}
% گروه آموزشی خود را وارد کنید
\degree {کارشناسی ارشد} 
% گروه آموزشی خود را وارد کنید
\subject{علوم کامپیوتر }
% گرایش خود را وارد کنید
\field{بیوانفورماتیک}
% عنوان پایان‌نامه را وارد کنید
\title{بررسی مسئله تکمیل شبکه بر روی شبکه تعاملات پروتئین-پروتئین}
% نام استاد(ان) راهنما را وارد کنید
\firstsupervisor{دکتر چنگیز اصلاحچی}
%\secondsupervisor{استاد راهنمای دوم}
% نام استاد(دان) مشاور را وارد کنید. چنانچه استاد مشاور ندارید، دستور پایین را غیرفعال کنید.
\firstadvisor{دکتر مهدی میرزایی}
%\secondadvisor{استاد مشاور دوم}
% نام پژوهشگر را وارد کنید
\name{مرتضی}
% نام خانوادگی پژوهشگر را وارد کنید
\surname{میلانی}
% تاریخ پایان‌نامه را وارد کنید
\thesisdate{\rl{دی ۱۳۹۳}}
% کلمات کلیدی پایان‌نامه را وارد کنید
\keywords{گراف، گرافلت، یادگیری ماشین، کرنل، بیوانفورماتیک، عملکرد پروتئین}
% چکیده پایان‌نامه را وارد کنید
\fa-abstract{\noindent
امروزه با پیشرفت تکنولوژی و گسترش دانش، بشر با تولید حجم عظیمی از داده روبرو است. بسیاری از این داده‌ها در قالب روابط بین موجودیت‌ها معنی پیدا می‌کنند. در بیولوژی داده‌هایی نظیر تعاملات پروتئین، روابط تنظیمی بین ژن‌ها، هم‌بیانی ژن‌ها، متابولیسم و ... در قالب گراف یا شبکه مدل می‌شوند. گراف، یک ساختار انعطاف‌پذیر و مناسب برای مدل‌سازی این نوع داده‌ها بدست می‌دهد. متأسفانه علیرغم اهمیت این موضوع، تکنیک‌های یادگیری ماشین روی ساختار گراف، دچار کمبود‌های جدی هستند و هنوز به تکامل نرسیده‌اند. دلیل این امر شاید پیچیده‌بودن مقایسه دو گراف باشد. در این پایان‌نامه، راه بهینه‌ای برای مقایسه دو گراف ارائه می‌دهیم و تابع کرنلی را تعریف می‌کنیم که به کمک آن می‌توان از الگوریتم‌های یادگیری ماشین روی داده‌های گراف به نحو بهینه و با دقت بالا استفاده کرد. سپس از این کرنل برای تکمیل شبکه‌های تعاملات پروتئین استفاده می‌کنیم. یعنی قسمت‌های دیده نشده از شبکه را پیش‌بینی و قسمت‌های اشتباه شبکه را تصحیح می‌کنیم.
}
\newpage
\thispagestyle{empty}
\vtitle
\newpage
\thispagestyle{empty}
\clearpage
~~~
\newpage
\thispagestyle{empty}
\vspace*{5cm}
{\dav
\begin{center}
كلية حقوق اعم از چاپ و تكثير، نسخه برداري ، ترجمه، اقتباس و ... از اين پايان نامه براي دانشگاه شهيد بهشتي محفوظ است.\\
 نقل  مطالب با ذكر مأخذ آزاد است.
\end{center}
}
\newpage
\thispagestyle{empty}
\clearpage
~~~
%%%%%%%%%%%%%%%%%%%%%%%%%%%%%%%%%%%%
\newpage
\begin{acknowledgementpage}

\vspace{4cm}

{\nastaliq
{\Large
تقدیم به\\
\vspace{2.5cm}
\hspace{2cm}
پدر و مادر عزیزم\\
\vspace{2.5cm}
\hspace{5cm}
به پاس گذشت‌‌هایشان\\
\vspace{2.5cm}
\hspace{2cm}
و همسرم\\
\vspace{2cm}
\hspace{5cm}
به پاس همراهی بی‌دریغش
}}
\end{acknowledgementpage}
\newpage
\thispagestyle{empty}
\clearpage
~~~
%%%%%%%%%%%%%%%%%%%%%%%%%%%%%%%%%%%%
\newpage
\thispagestyle{empty}
{\nastaliq
تشکر و قدردانی
}
\\[2cm]
نخست، پروردگارم را می‌ستایم به پاس رحمت‌های بی منتهایش. و سپس، پدر و مادرم را گرامی می‌دارم به پاس صبوریشان و همسرم را عزیز می‌شمارم به پاس همراهیش.

از استاد ارجمندم، جناب آقای دکتر چنگیز اصلاحچی که با حسن خلق و فروتنی، از هیچ کمکی دریغ ننمودند و در این راه من را راهنمایی کردند، تشکر می‌کنم. تنش به ناز طبیبان بی‌نیاز باد.

از جناب آقای دکتر مهدی میرزایی که در انجام این طرح از کمک‌هایشان استفاده کردم تشکر می‌کنم.

\signature 
\newpage
\thispagestyle{empty}
\clearpage
~~~
\newpage
%{\small
\abstractview
%}
\newpage
\thispagestyle{empty}
\clearpage
~~~
\newpage